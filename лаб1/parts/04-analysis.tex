\chapter{Теоретическая часть}
В данной части рассмотрены формулы для вычисления величин, эмпирическая плотность, эмпирическая функция распределения и гистограмма.

\section{Формулы для вычисления величин}
\textbf{Минимальное значение выборки}
\begin{equation}
    M_{min} = min\{x_1,\cdots,x_n\},
\end{equation}
 где $(x1,\cdots,x_n)$ -- реализация случайной выборки.

 \textbf{Максимальное значение выборки}
 \begin{equation}
     M_{max} = max\{x_1,\cdots,x_n\},
 \end{equation}
 где $(x1,\cdots,x_n)$ -- реализация случайной выборки.

 \textbf{Размах выборки}
 \begin{equation}
     R = M_{max} - M_{min},
 \end{equation}
где $M_{max}$ -- максимальное значение выборки, $M_{min}$ --минимальное значение выборки.

\textbf{Оценка математического ожидания}
\begin{equation}
    \hat{\mu}(\vec{X_n}) = \bar{X_n}=\frac{1}{n}\sum_{i=1}^n X_i
\end{equation}

\textbf{Несмещённая оценка дисперсии}
\begin{equation}
    S^2(\vec{X_n}) = \frac{n}{n-1} \hat{\sigma} = \frac{n}{n-1}\sum_{i=1}^n (X_i - \bar{X_n})^2
\end{equation}

\textbf{Выборочная дисперсия}
\begin{equation}
    \hat{\sigma}(\vec{X_n}) = \frac{1}{n}\sum_{i=1}^n(X_i - \bar{X_n})^2
\end{equation}

\textbf{Количество интервалов}
\begin{equation}
    m = [\log_{2}n] + 2
\end{equation}

\section{Эмпирическая плотность и гистограмма}
\textit{Определение.} Эмпирической плотностью распределения случайной
выборки $\Vec{X_n}$ называют функцию:

\begin{equation}
    f_n(x) = \begin{cases}
        \frac{n_i}{n\Delta}, x \in J_i, i = \overline{1, m}; \\
        0, \textit{иначе}.
    \end{cases}
\end{equation}
где $J_i, i = \overline{1, m}$ --  полуинтервал из $J=[x_{(1)}, x_{(n)}]$.

\begin{equation*}
\begin{split}
    &x_{(1)}=min\{x_1,\cdots,x_n\},\\
    &x_{(n)}=max\{x_1, \cdots, x_n\},\\
    &J_i=[x_{(1)}+(i-1)\Delta], x_{(i)}+i\Delta, i = \overline{1, m-1},\\
    &J_m=[x_{(1)}+(m-1)\Delta, x_{(1)}+m\Delta],\\
\end{split}
\end{equation*}
$m$ -- количество полуинтервалов интервала $J=[x_{(1)}, x_{(n)}]$,\newline
$\Delta$ -- длина полуинтервала $J_{i}, i=\bar{1, m}$ равная 

\begin{equation}
    \Delta=\frac{x_{(n)} - x_{(1)}}{m}=\frac{|J|}{m}
\end{equation}\newline
$n_{i}$ -- количество элементов выборки в полуинтервале $J_{i}, i = \overline{1, m}$,\newline
$n$ -- количество элементов в выборке.

\textit{Определение.} График функции $f_n(x)$ называют гистограммой.

\section{Эмпирическая функция распределения}
\textit{Определение.}Пусть
\begin{enumerate}
    \item $\vec{X_n}=(X_1,\cdots,X_n)$ -- случайная выборка,
    \item $\vec{x_n}=(x_1,\cdots,x_n)$ -- реализация случайной выборки,
    \item $n(x,\vec{x_n})$ -- количество элементов выборки $\vec{x_n}$, которые меньше $x$, тогда эмпирической функцией распределения называют функцию 
\end{enumerate}

\begin{equation}
    F_n:\mathbb{R}\rightarrow \mathbb{R},\quad F_n=\frac{n(x, \vec{x_n})}{n}.
\end{equation}

\textit{Замечание.}
\begin{enumerate}
    \item $F_n(X)$ обладает всеми свойствами функции распределения;
    \item $F_n(X)$ кусочно-постоянна;
    \item если все элементы вектора различны, то
    \begin{equation}
    F_n(x) = \begin{cases}
        0, & \quad x \leq x_{(1)};\\
        \frac{i}{n}, & \quad x_{(i)} < x \leq x_{(i+1)}, \quad i=\bar{1, n-1};\\
        1, & \quad x > x_{(n)}
    \end{cases}
\end{equation}
    \item Эмпирическая функция распределения позволяет интерпретировать выборку $\vec{x_n}$ как реализацию дискретной случайной величины $\widetilde{X}$, ряд распределения которой имеет вид:

   \begin{center}
    \begin{tabularx}{0.8\textwidth}{|>{\centering\arraybackslash}X|>{\centering\arraybackslash}X|>{\centering\arraybackslash}X|>{\centering\arraybackslash}X|}
        $\widetilde{X}$ & $x_{(1)}$ & $\ldots$ & $x_{(n)}$ \\
        \hline
        $P$ & $\frac{1}{n}$ & $\ldots$ & $\frac{1}{n}$ \\ 
    \end{tabularx}
\end{center}
\end{enumerate}

Это позволяет рассматривать числовые характеристики случайной величины $\widetilde{X}$ как приближенные значения числовых характеристик случайной величины $X$.
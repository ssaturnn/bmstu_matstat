\chapter{Теоретическая часть}

\section{Определение $\gamma$~-~доверительного интервала для значения параметра распределения случайной величины}

Пусть:
1) $X$ -- случайная величина, закон распределения которой
известен с точностью до неизвестного параметра $\theta$.

Закон распределения с.в. $X$ известен с точностью до $\theta$
означает, что известен общий вид закона распределения с.в. $X$, но этот закон
зависит от неизвестного параметра $\theta$.
Если задать некоторое значение $\theta$, 
то закон распределения с.в. X будет известен полностью

Интервальной оценкой с уровнем доверия $\gamma$ 
($\gamma$~-~доверительной интервальной оценкой) 
параметра $\theta$ называют пару статистик 
$\underline{\theta}(\vec X), \overline{\theta}(\vec X)$ таких, что

\begin{equation*}
    P\{\underline{\theta}(\vec X)<\theta<\overline{\theta}(\vec X)\}=\gamma
\end{equation*}

% Поскольку границы интервала являются случайными величинами, то для различных реализаций случайной выборки $\vec X$ статистики $\underline{\theta}(\vec X), \overline{\theta}(\vec X)$ могут принимать различные значения.

Доверительным интервалом с уровнем доверия $\gamma$ ($\gamma$-доверительным интервалом) называют интервал $(\underline{\theta}(\vec x), \overline{\theta}(\vec x))$, отвечающий выборочным значениям статистик $\underline{\theta}(\vec X), \overline{\theta}(\vec X)$.

\section{Формулы для вычисления границ \\ $\gamma$-доверительного интервала для математического ожидания и дисперсии нормальной случайной величины}

Статистику $\hat{\theta}(\vec{X})$ называют точечной оценкой параметра $\theta$,
если ее выборочное значение принимается в качестве параметра $\theta$.
Т.е. $\theta = \hat{\theta}(\vec{x})$ 

Формулы для вычисления границ $\gamma$-доверительного интервала для математического ожидания:

\begin{equation}
\underline\mu(\vec X_n)=\overline X + \frac{S(\vec X)t^{St(n-1)}_{\frac{1-\gamma}{2}}}{\sqrt{n}}
\end{equation}

\begin{equation}
\overline\mu(\vec X_n)=\overline X + \frac{S(\vec X)t^{St(n-1)}_{\frac{1+\gamma}{2}}}{\sqrt{n}}
\end{equation}

$\overline X$ -- точечная оценка математического ожидания;
$S(\vec X) = \sqrtsign{S^2(\vec X)}$ -- квадратный корень из точечной оценки дисперсии;
$n$ -- объем выборки;
$\gamma$ -- уровень доверия;
$t^{St(n-1)}_{\alpha}$ -- квантиль уровня $\alpha$ распределения Стьюдента с $n - 1$ степенями свободы.

Формулы для вычисления границ $\gamma$-доверительного интервала для дисперсии:

\begin{equation}
\underline\sigma(\vec X_n)= \frac{(n-1)S^2(\vec X)}{t^{\chi^2(n-1)}_{\frac{1+\gamma}{2}}}
\end{equation}

\begin{equation}
\overline\sigma(\vec X_n)= \frac{(n-1)S^2(\vec X)}{t^{\chi^2(n-1)}_{\frac{1-\gamma}{2}}}
\end{equation}

$S^2(\vec X)$ -- точечная оценка дисперсии;
$n$ -- объем выборки;
$\gamma$ -- уровень доверия;
$t^{\chi^2(n-1)}_{\alpha}$ -- квантиль уровня $\alpha$ распределения $\chi^2(n-1)$ с $n - 1$ степенями свободы.